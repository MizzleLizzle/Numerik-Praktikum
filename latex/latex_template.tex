\documentclass[a4paper,12pt]{article}
%\documentclass[a4paper,12pt,twoside]{book}

\usepackage[margin=3.5cm]{geometry}

\usepackage[utf8x]{inputenc}
\usepackage[ngerman]{babel}
\usepackage{longtable}
\usepackage{amsmath}
\usepackage{amsfonts}
\usepackage{amssymb}
\usepackage{amsthm}
\usepackage{graphicx}
\usepackage{tikz}
\usepackage{epsfig}




\newtheorem{theorem}{Satz}
\newtheorem{lemma}[theorem]{Lemma}

\theoremstyle{definition}
\newtheorem{definition}[theorem]{Definition}
\newtheorem{bemerkung}[theorem]{Bemerkung}


% % if document type is book, the following commands lead to an enumeration 
% % of theorems etc of the type 2.3, if it is the third theorem in chapter 2
% \numberwithin{theorem}{chapter}
% \numberwithin{figure}{chapter}
% \numberwithin{equation}{chapter}



\newcommand{\R}{\mathbb{R}}






\begin{document}

\begin{titlepage}\parindent0cm
Universit\"at Leipzig\\
Fakult\"at f\"ur Mathematik und Informatik\\
Mathematisches Insitut\\

\vspace{5cm}


\begin{center}
\huge{PROJEKTTITEL}

\vspace{1cm}
\Large{Projekt im Numerischen Praktikum}\\
\vspace{0.5cm}
\large{eingereicht von NAMEN}
\end{center}

\vfill
\begin{minipage}{\textwidth}
Leipzig, den DATUM
\end{minipage}
\end{titlepage}




% % if documentclass is book, \frontmatter leads to an enumeration like i, ii, iii 
% % of the pages
% \frontmatter

\tableofcontents

% % if documentclass is book, \mainmatter starts the standard page numbering
% % of the pages
% \mainmatter

\newpage


\section{Einleitung}


\section{Erster Abschnitt} 

\subsection{Erster Unterabschnitt}


\subsection{Hier z.B.\ numerische Beispiele}

Dieser Abschnitt enth\"alt eine Umgebung f\"ur eine Abbildung 
in~\ref{f:Bezeichnung} und die Tabelle~\ref{tab:Bezeichnung}.

\begin{figure}
    %\includegraphics[width=0.8\textwidth]{name-des-Bildes}
    \caption{\label{f:Bezeichnung}Hier kann eine Abbildung hinein.}
\end{figure}

\begin{table}
    \begin{tabular}{l|ll}
        $k$ & $x_k$ & $y_k$\\
        \hline 
        $1$ & $2$ & $2.5$ \\
        $2$ & $z_k$ & $3.654$
    \end{tabular}
    \caption{\label{tab:Bezeichnung}Dies ist eine Beispiel-Tabelle.}
\end{table}


Au{\ss}erdem ist hier noch die Definition~\ref{def:newton} und der 
Satz~\ref{th:conv_newton} enthalten.

\begin{definition}[Newton-Verfahren]\label{def:newton}
    Das Newton-Verfahren ist definiert durch\dots
\end{definition}


\begin{theorem}[Konvergenz des Newton-Verfahrens]\label{th:conv_newton}
    Das Newton-Verfahren ist lokal quadratisch konvergent.
\end{theorem}

\begin{proof}
    Hier ist der Beweis. Siehe auch \cite[(5.3.7) Theorem]{FreundHoppe2007}.
\end{proof}


\section{Evtl.\ Zusammenfassung}



%\newpage
\addcontentsline{toc}{section}{Literaturverzeichnis}
\bibliographystyle{alpha}
\bibliography{ref.bib}

% % if documentclass is book, \backmatter causes that chapters etc are not 
% % enumerated
% \backmatter

\newpage 

\noindent
{\Large Erkl\"arung}\bigskip\\

\noindent
Ich/Wir erkl\"aren, dass ich/wir die vorliegende Arbeit selbstst\"andig 
und nur unter Verwendung der angegebenen Literatur und Hilfsmittel angefertigt 
habe/haben.\bigskip\\

\noindent
Leipzig, den \today 

\addcontentsline{toc}{section}{Selbstst\"andigkeitserkl\"arung}


\end{document}




